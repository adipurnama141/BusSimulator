\documentclass{article}
\usepackage{amsmath}
\usepackage{listings}
\usepackage{hyperref}

\title{Simulasi Event Diskrit (Studi Kasus : Simulasi Sistem Transportasi Bus)}
\date{1-03-2018}
\author{Adi Purnama}

\author{
  Adi Purnama\\
  \texttt{13514006@std.stei.itb.ac.id}
  \and
  Hafizh Afkar Makmur\\
  \texttt{13514062@std.stei.itb.ac.id}
}


\lstset{
				breaklines,
    tabsize=1
}



\begin{document}
	\maketitle
	\pagenumbering{arabic}

	\section{Deskripsi Masalah}	
Sebuah bus memiliki tiga titik pemberhentian, yaitu Air Terminal 1 , Air Terminal 2 , dan Car Rental 3.
Bus memulai rute dari pemberhentian 3 , lalu ke 2, ke 1 , kemudian kembali ke 3 , dan begitu seterusnya (loop). 
Bus memiliki kapasitas 20 penumpang. Kecepatan bus adalah 30 mil per jam, dengan jarak antar titik pemberhentian Car Rental ke Air Terminal 1 sejauh 4.5 Mil, Air Terminal 1 ke Air Terminal 2 sejauh 1 mil , dan Air Terminal 2 ke Car Rental 3 sejauh 3.5 Mil.
Calon penumpang data ke setiap titik pemberhentian dengan distribusi independent exponential interarrival time. Rata – rata kedatangan calon penumpang per jam pada masing – masing titik pemberhentian untuk Air Terminal 1 adalah 14 orang per jam, Air Terminal 2 adalah 10 orang per jam, dan Car Rental 3 adalah 24 orang per jam.

Penumpang yang berasal dari Air Terminal memiliki tujuan ke Car Rental. Sementara itu, penumpang yang berasal dari Car Rental memiliki tujuan ke Air Terminal 1 atau 2 dengan probabilitas tujuan air terminal 1	adalah 0.583 dan probabilitas tujuan air terminal 2 adalah 0.417.
Aturan antrian (baik itu antrian di titik pemberhentian, maupun antrian saat ingin keluar dari bus) adalah First In First Out. Bus akan menunggu di masing-masing titik pemberhentian selama minimal 5 menit. Setelah itu , bus akan berangkat.
Waktu yang diperlukan penumpang untuk memasuki bus dan keluar bus bervariasi setiap orangnya. 
Waktu untuk turun dari bus memiliki distribusi uniform antara 16 sampai 24 detik. Sementara itu, waktu untuk masuk ke bus memiliki distribusi uniform antara 15 sampai 25 detik.

Jalankan simulasi sampai 80 jam. Lalu berikan laporan statistik sebagai berikut :
	\begin{enumerate}
	\item Panjang Antrian di tiap titik pemberhentian (rata rata \& maksimum)
	\item Waktu tunggu yang dirasakan penumpang saat menunggu bis datang di titik pemberhentian (rata rata \& maksimum)
	\item Jumlah penumpang yang ada di dalam bis (rata rata \& maksimum )
	\item Waktu tunggu bis di titik pemberhentian (rata rata, maksimum, minimum)
	\item Waktu bis melakukan 1 loop. 1 Loop didefinisikan sejak keberangkatan dari Car Rental (rata rata , maksimum, minimum)
	\item Waktu total perjalanan yang dirasakan penumpang. Waktu total dimulai sejak penumpang datang ke titik pemberhentian sampai keluar dari bis. (rata rata \& maksimum)
	\end{enumerate}


	\section{Rancangan Solusi}
	Untuk menyelesaikan permasalahan ini, dilakukan dua tahap pengerjaan.
	Langkah pertama adalah mensimulasikan proses sesuai dengan deskripsi di atas. Setelah proses tersebut dapat disimulasikan menggunakan program komputer, dilakukan langkah selanjutnya yaitu perhitungan data statistik. 
	Berikut adalah komponen utama dari program simulasi ini.


	\subsection{Kelas Passanger}
	Kelas untuk menyimpan informasi yang dimiliki oleh seorang penumpang, seperti waktu kedatangan , waktu keberangkatan, kota asal dan kota tujuan. Kelas passanger memiliki dua buah operasi untuk keluar dan masuk bis yaitu enteringTheBus() dan leavingTheBus()
	\begin{lstlisting}[language=Python]
	class Passanger:
	def __init__(self,id,originStation,arrivedTimeAtOriginStation):
		self.id = id
		self.originStation = originStation
		self.targetStation = 0
		self.arrivedTimeAtOriginStation = arrivedTimeAtOriginStation
		self.arrivedTimeAtTargetStation = 0
		self.timeEnteringTheBus = 0
		self.timeLeavingTheBus = 0
		self.delayTimeInQueue = 0
	def enteringTheBus(self, timeEnteringTheBus):
		self.timeEnteringTheBus = timeEnteringTheBus
		self.delayTimeInQueue = self.timeEnteringTheBus - self.arrivedTimeAtOriginStation
	def leavingTheBus(self, timeLeavingTheBus):
		self.timeLeavingTheBus = timeLeavingTheBus
	\end{lstlisting}


	\pagebreak

	\subsection{Kelas StationQueue}
	Kelas yang merepresentasikan sebuah stasiun pemberhentian bis. Komponen yang menyusun kelas ini antara lain :

\subsubsection{futurePassangerQueue}
List yang berisi seluruh calon penumpang yang akan datang ke stasiun pemberhentian ini sejak awal sampai akhir waktu simulasi. List ini di-generate saat awal simulasi (method initializeAllFutureEvents()), menggunakan waktu antar kedatangan dengan distribusi eksponensial.

\subsubsection{simulationTime}
Menyimpan waktu simulasi saat ini

\subsubsection{currentQueue}
List yang berisi seluruh calon penumpang yang sedang menunggu antrian pada waktu simulationTime. List ini diupdate menggunakan method updateStationQueue dengan cara memindahkan penumpang dari futurePassangerQueue ke currentQueue berdasarkan waktu simulasi saat ini. Method updateStationQueue dipanggil setiap kali waktu simulasi bergerak maju.


	\begin{lstlisting}[language=Python]

class StationQueue:
	def __init__(self,passangerPerHour,originStation):
		self.futurePassangerQueue = []
		self.currentQueue = []
		self.simulationTime = 0
		self.isFirstStart = True
		self.maxNumberQueue = 0
		self.qtArea_nPeopleOnQueue = 0
		self.initializeAllFutureEvents(passangerPerHour,originStation)

	def initializeAllFutureEvents(self,passangerPerHour,originStation):
		global_time = 0.0
		simulation_end = 81
		simulation_end = simulation_end * 3600
		n_passanger = 0
		stationList = []
		while global_time < simulation_end:
			passanger_arrival = random.expovariate(passangerPerHour) * 3600
			global_time += passanger_arrival
			if (global_time <= simulation_end):
				n_passanger += 1
				a = Passanger(n_passanger,originStation,global_time)
				self.futurePassangerQueue.append(a)
	
	def updateStationQueue(self,time):
		if (self.isFirstStart):
			self.simulationTime = 0
			self.isFirstStart = False
		else:
			while (self.futurePassangerQueue[0].arrivedTimeAtOriginStation <= time):
				passanger = self.futurePassangerQueue.pop(0)
				self.currentQueue.append(passanger)
			self.qtArea_nPeopleOnQueue += len(self.currentQueue) * (time - self.simulationTime)
			self.simulationTime = time
			if (self.maxNumberQueue < len(self.currentQueue)):
				self.maxNumberQueue = len(self.currentQueue)
	
	def peekFuturePassanger(self):
		return self.futurePassangerQueue[0]

\end{lstlisting}


	\subsection{Kelas Bus}
Kelas untuk merepresentasikan sebuah Bus. Program utama dari simulasi ini adalah sebagai berikut
\begin{lstlisting}[language=Python]
bus = Bus()
while True:
	bus.unloadPassangers()
	bus.loadPassangers()
	bus.movePlace()
\end{lstlisting}

\subsubsection{unloadPassangers()}
Method untuk menurunkan seluruh penumpang yang ingin turun dari Bus.
\begin{lstlisting}[language=Python]
def unloadPassangers(self):
	if (self.nOfPassangerWantToUnloadHere > 0):
		for x in range(self.nOfPassangerWantToUnloadHere):
			passangerWantToUnload = self.passangersInsideBus.pop(0)
			self.unloadPassanger(passangerWantToUnload)
\end{lstlisting}

\subsubsection{unloadPassanger()}
Method untuk menurunkan seorang penumpang yang ingin turun dari Bus. Method unloadPassangerDuration() berfungsi untuk men-generate uniform random number antara 16 sampai 24 detik.
\begin{lstlisting}[language=Python]
def unloadPassanger(self,passanger):
		unloadTime = self.unloadPassangerDuration()
		passanger.timeArrivedAtStation = self.simulationTime
		self.simulationTime += unloadTime
		passanger.timeLeavingTheBus = self.simulationTime
		passanger.targetStation = self.currentPosition
		self.arrivedPassangers.append(passanger)	
		if (passanger.originStation == 3):
			self.nOfPassangerFromCarRental -= 1
		else:
			self.nOfPassangerFromTerminal -= 1
		self.qtArea_nPassangerInsideBus = self.qtArea_nPassangerInsideBus + (self.simulationTime - passanger.timeEnteringTheBus)
		self.timeoutCheck()
\end{lstlisting}

\subsubsection{loadPassangers()}
Method untuk menunggu dan memasukan penumpang yang ingin naik ke Bus. Algoritma yang digunakan adalah sebagai berikut

\begin{enumerate}
	\item Naikkan seluruh penumpang yang saat ini sudah menunggu di stasiun
	\item Ramalkan kedatangan penumpang di masa depan (panggil method peekFuturePassanger() untuk mendapatkan penumpang teratas di queue futurePassangerQueue). Jika penumpang tersebut akan datang sebelum batas waktu tunggu berakhir. Tunggu hingga penumpang itu datang. Jika tidak, lanjutkan perjalanan ke stasiun berikutnya.
\end{enumerate}


\begin{lstlisting}[language=Python]
def loadPassangers(self):
		currentSessionbusWaitingTime = self.getMaxbusWaitingTime()
		currentStationQueue = self.StationQueue[self.currentPosition]
		isWaitingPhase = True
		while (self.simulationTime <= currentSessionbusWaitingTime) and (not self.isFull()) and (isWaitingPhase):

			# Load All Passanger Who Already Waiting in Station Queue as Long Bus is Not Full
			while (len(currentStationQueue.currentQueue) > 0) and (not self.isFull()):
				passanger = currentStationQueue.currentQueue.pop(0)
				self.loadPassanger(passanger)
				# Additional Waiting Time Due People Loading
				if (self.simulationTime > currentSessionbusWaitingTime):
					currentSessionbusWaitingTime = self.simulationTime
			
			# Peek Future Passanger. If There is Future Passanger Who Will Come Faster Before WaitTime Ends. Wait. If No. Wait Until WaitTime Ends. Then Go..
			if (currentStationQueue.peekFuturePassanger().arrivedTimeAtOriginStation <= currentSessionbusWaitingTime):
				self.setsimulationTime(currentStationQueue.peekFuturePassanger().arrivedTimeAtOriginStation)
			else:
				isWaitingPhase = False
				
				# Force Wait Until 5 Minutes (Even Is Predicted There Is No Passanger Will Come Anyway)
				if (self.simulationTime < currentSessionbusWaitingTime):
					self.setsimulationTime(currentSessionbusWaitingTime)

				# Collect The Statistics Data (Waiting Time Duration)	
				busWaitingTime = self.simulationTime - self.timeArrivedAtStation
				self.addbusWaitingTimeStatistics(busWaitingTime)
				
				# Collect The Statistics Data (Loop Time)
				if ((self.currentPosition == 3) and (self.isFreshStart)):
					self.carRentalDepartureTime = self.simulationTime
				if ((self.currentPosition == 3) and (not self.isFreshStart)):
					loopTime = self.simulationTime - self.carRentalDepartureTime
					self.carRentalDepartureTime = self.simulationTime
					self.addLoopTimeStatistics(loopTime)

\end{lstlisting}


\pagebreak
\subsubsection{loadPassanger()}
Method untuk menaikan satu penumpang dari stasiun ke dalam bis.

\begin{lstlisting}[language=Python]
def loadPassanger(self,passanger):
	loadTime = self.loadPassangerDuration()
	passanger.enteringTheBus(self.simulationTime + loadTime)
	self.passangersInsideBus.append(passanger)
	self.simulationTime = passanger.timeEnteringTheBus
	if (passanger.originStation == 3):
		self.nOfPassangerFromCarRental += 1
	else:
		self.nOfPassangerFromTerminal +=1
	self.timeoutCheck()
\end{lstlisting}

\subsubsection{movePlace()}
Method yang berfungsi untuk memindahkan bus dari satu stasiun ke stasiun selanjutnya. 
\begin{lstlisting}[language=Python]
def movePlace(self):
		self.isFreshStart = False	
		# [STATS] Check if Current Number of People in The Bus Breaks Record
		if (self.maxNPassangers < len(self.passangersInsideBus)):
			self.maxNPassangers = len(self.passangersInsideBus)
		if (self.currentPosition == 3):
			self.currentPosition = 1
			self.simulationTime = self.simulationTime + (4.5 / self.busSpeed * 3600)
			self.determinePassangerTargetStation()
			self.nOfPassangerWantToUnloadHere = self.nOfPassangerToTerminal1
			self.nOfPassangerFromTerminal = 0
		elif (self.currentPosition == 1):
			self.currentPosition = 2
			self.simulationTime = self.simulationTime + (1 / self.busSpeed * 3600)
			self.nOfPassangerWantToUnloadHere = self.nOfPassangerToTerminal2
			self.nOfPassangerToTerminal1 = 0
		else:
			self.currentPosition = 3
			self.simulationTime = self.simulationTime + (4.5 / self.busSpeed * 3600)
			self.nOfPassangerWantToUnloadHere = self.nOfPassangerFromTerminal
			self.nOfPassangerToTerminal2 = 0
		self.timeArrivedAtStation = self.simulationTime
		self.timeoutCheck()
\end{lstlisting}

\subsubsection{determinePassangerTargetStation()}
Saat melakukan perpindahan, bus juga mencatat tujuan perjalanan dari masing-masing penumpang.
Seluruh penumpang yang berasal dari Air Terminal memiliki tujuan ke Car Rental.
Sementara itu, penumpang dari Car Rental dapat memiliki tujuan ke Air Terminal 1 atau Air Terminal 2 berdasarkan
probabilitas tertentu. Probabilitas ini dihitung menggunakan method determinePassangerTargetStation()
\begin{lstlisting}[language=Python]
def determinePassangerTargetStation(self):
		for x in range(self.nOfPassangerFromCarRental):
			if (random.uniform(0,1) <= 0.583):
				self.nOfPassangerToTerminal1 += 1
			else:
				self.nOfPassangerToTerminal2 += 1
\end{lstlisting}

\subsubsection{timeoutCheck()}
Method timeoutCheck() dipanggil setiap kali waktu simulasi bergerak maju. Method ini berfungsi untuk melakukan refresh terhadap antrian pada tiap stasiun dan mengakhiri simulasi apabila waktu simulasi sudah melebihi batas (80 Jam)
\begin{lstlisting}[language=Python]
def timeoutCheck(self):
		self.refreshQueue()
		if self.simulationTime >= (80 *3600):
			self.printStatistics()
			sys.exit()
\end{lstlisting}

\pagebreak
\subsection{Statistik}
Secara umum, perhitungan statistik di dalam kasus ini dapat digolongkan menjadi dua jenis

\subsubsection{Rata-rata, maksimum, minimum}
Data tersebut dikumpulkan dalam sebuah struktur data list, lalu dihitung maksimum, minimum , dan rata-ratanya.
Sebagai contoh , waktu tunggu yang dirasakan penumpang saat menunggu bis datang di titik pemberhentian (rata rata \& maksimum).
Langkah menghitung statistik ini adalah sebagai berikut :
\begin{enumerate}
\item Setiap penumpang memiliki informasi "Waktu menunggu di stasiun". Kumpulkan informasi ini dari seluruh penumpang, lalu masukkan ke dalam sebuah struktur data list.
\item Lakukan operasi avg() dan max() terhadap struktur data list tersebut.
\end{enumerate}
Sebagian besar perhitungan statistik yang dapat diselesaikan dengan cara ini, antara lain :
	\begin{enumerate}
	\item Panjang maksimum antrian di tiap titik pemberhentian 
	\item Waktu tunggu yang dirasakan penumpang saat menunggu bis datang di titik pemberhentian (rata rata \& maksimum)
	\item Jumlah maksimum penumpang yang ada di dalam bis 
	\item Waktu tunggu bis di titik pemberhentian (rata rata, maksimum, minimum)
	\item Waktu bis melakukan 1 loop. 1 Loop didefinisikan sejak keberangkatan dari Car Rental (rata rata , maksimum, minimum)
	\item Waktu total perjalanan yang dirasakan penumpang. Waktu total dimulai sejak penumpang datang ke titik pemberhentian sampai keluar dari bis. (rata rata \& maksimum)
	\end{enumerate}
	Semua perhitungan statistik di atas mengikuti pola yang sama, yaitu kumpulkan seluruh informasi ke dalam sebuah list , lalu operasikan list tersebut dengan fungsi avg(), min(), atau max(). Berikut adalah potongan source code untuk menghitung statistik.
\begin{lstlisting}[language=Python]
def printLoopTimeStatistics(self):
		l = self.loopTimeStatistics
		avgLoopTime = avg(l)
		maxLoopTime = max(l)
		minLoopTime = min(l)
		print "Loop Time"
		print "	Max: ",
		timeFormatter(maxLoopTime)
		print "	Avg: ",
		timeFormatter(avgLoopTime)			
		print "	Min: ",
		timeFormatter(minLoopTime)
\end{lstlisting}

\subsubsection{Rata Rata Panjang Antrian}
Statistik mengenai rata-rata panjang antrian dihitung menggunakan langkah sebagai berikut.
\begin{enumerate}
\item Untuk setiap panjang antrian, hitung durasi terjadinya
\item Jumlahkan hasil kali panjang antrian dengan durasi terjadinya
\item Bagi hasil penjumlahan tersebut dengan waktu akhir simulasi
\end{enumerate}
Sebagai contoh, perhatikan tabel di bawah ini :

\begin{center}
 \begin{tabular}{||c | c||} 
 \hline
 Panjang Antrian & Durasi Waktu Terjadi (s) \\ [0.5ex] 
 \hline\hline
 1 & 2.3 \\ 
 \hline
 2 & 7.1 \\
 \hline
 3 & 2.3 \\
 \hline
\end{tabular}
\end{center}

Rata-rata panjang antrian dapat dihitung sebagai berikut

$$ rataRataPanjangAntrian = \frac{(1x2.3) + (2x7.1) + (3 x 2.3)}{2.3 + 7.1 + 2.3}  $$

Pada kasus ini, terdapat dua perhitungan statistik yang melibatkan rata-rata panjang antrian, yaitu :
\begin{enumerate}
	\item Rata-rata panjang antrian di setiap stasiun pemberhentian
	\item Rata-rata jumlah penumpang yang menaiki bus
\end{enumerate}

Untuk rata-rata panjang antrian di setiap stasiun pemberhentian, jumlahkan seluruh durasi waktu menunggu yang dialami oleh setiap penumpang pada suatu stasiun, lalu bagi dengan waktu akhir simulasi. Total seluruh waktu menunggu yang dialami oleh penumpang pada suatu stasiun disimpan pada atribut qtAreanPeopleOnQueue di kelas StationQueue. Setiap kali waktu simulasi dimajukan sebesar a detik, qtAreanPeopleOnQueue akan ditambah dengan (a x jumlahOrangdiAntrianSaatIni)
\begin{lstlisting}[language=Python]
def updateStationQueue(self,time):
			self.qtArea_nPeopleOnQueue += len(self.currentQueue) * (time - self.simulationTime)
\end{lstlisting}
Lalu, bagi nilai itu dengan waktu akhir simulasi untuk mendapatkan rata-rata panjang antrian.
\begin{lstlisting}[language=Python]
def printNumberPersonQueueStatistics(self):
	print "Average Number Queue"
	print "	Station1 :",
	print self.StationQueue[1].qtArea_nPeopleOnQueue / self.simulationTime
	print "	Station2 :",
	print self.StationQueue[2].qtArea_nPeopleOnQueue / self.simulationTime
	print "	Station3 :",
	print self.StationQueue[3].qtArea_nPeopleOnQueue / self.simulationTime
\end{lstlisting}
Untuk rata-rata jumlah penumpang yang menaiki bus, jumlahkan seluruh durasi waktu selama di dalam bis yang dialami oleh setiap penumpang, lalu bagi dengan waktu akhir simulasi.


\section{Hasil Simulasi}

\subsection{Statistik Bus}

\begin{center}
 \begin{tabular}{||c | c | c | c||} 
 \hline
 Statistik & Max & Avg & Min \\ [0.5ex] 
 \hline\hline
 Bus Waiting Time & 0:13:43 & 0:09:17 & 0:05:00 \\ 
 \hline
 Bus Loop Time & 0:48:17 & 0:45:15 & 0:38:12 \\
 \hline
 Number of People On The Bus & 20 & 13.9710474206 & - \\
 \hline
\end{tabular}
\end{center}


Hal menarik yang dapat disimpulkan dari hasil statistik ini adalah sebagai berikut :

\begin{enumerate}
	\item Rata-rata, bus menunggu di stasiun selama sekitar 9 menit
	\item Rata-rata, waktu yang diperlukan bus untuk melakukan satu putaran adalah 45 menit
	\item Rata-rata, bus selalu terisi 14 orang dari total kapasitas 20 (Utilisasi 69\%)
\end{enumerate}  

\pagebreak
\subsection{Statistik Stasiun}

\begin{center}
 \begin{tabular}{||c | c | c | c||} 
 \hline
 Statistik & Air Terminal 1 & Air Terminal 2 & Car Rental \\ [0.5ex] 
 \hline\hline
 Average Number Queue & 6.18643751187 & 7.70652576188 & 9.31329665976 \\ 
 \hline
 Avg Delay Time & 0:25:04 & 0:45:50 & 0:23:13 \\
 \hline
 Max Delay Time & 1:16:40 & 3:40:37 & 1:08:15 \\
 \hline
 Max Number Queue & 22 & 39 & 28 \\
 \hline
\end{tabular}
\end{center}


Hal menarik yang dapat disimpulkan dari hasil statistik ini adalah sebagai berikut :

\begin{enumerate}
	\item Rata-rata , stasiun Car Rental cenderung lebih ramai dibanding stasiun yang lain dengan panjang antrian rata-rata 9 orang. 
	\item Dapat disimpulkan bahwa kualitas layanan terburuk ada di Stasiun Air Terminal 2. Penumpang akan merasakan waktu menunggu lebih lama jika dia menunggu di Stasiun Air Terminal 2 (45 Menit). Pada kasus terburuk, penumpang dapat menunggu hingga 3 jam 40 menit untuk menunggu bus di Stasiun Air Terminal 2. Hal ini dapat disebabkan karena bus yang datang sudah penuh, sehingga harus menunggu bus untuk melakukan satu putaran lagi. Mengingat waktu rata-rata bus untuk melakukan 1 putaran adalah 45 menit, dalam kasus ini, penumpang harus menunggu bis hingga 4 putaran karena bus tersebut selalu penuh. Pada kasus terburuk, panjang antrian dapat mengular hingga 39 orang, kejadian ini juga terjadi di Stasiun Air Terminal 2.
	\item Meskipun Stasiun Air Terminal 2 cenderung sepi dikunjungi (10 orang/jam), kualitas layanan di stasiun ini cenderung paling buruk. Hal ini disebabkan karena ramainya calon penumpang dari Stasiun Air Terminal 1 (14 orang/jam) dan Stasiun Car Rental (24 orang/jam). Sebelum ke Stasiun Air Terminal 2, bus harus melalui Stasiun Car Rental \& Stasiun Air Terminal 1 . Saat bus sampai di Stasiun Air Terminal 2, kemungkinan bus sudah penuh. Hal ini mengakibatkan calon penumpang harus menunggu bus selanjutnya datang. 
\end{enumerate}  


\subsection{Statistik Penumpang}

\begin{center}
 \begin{tabular}{||c | c | c | c||} 
 \hline
 Statistik & Max & Avg & Min \\ [0.5ex] 
 \hline\hline
 Passanger In System Duration & 3:56:19 & 0:46:48 & 0:12:54 \\ 
 \hline
\end{tabular}
\end{center}


Hal menarik yang dapat disimpulkan dari hasil statistik ini adalah sebagai berikut :

\begin{enumerate}
	\item Rata-rata, penumpang menghabiskan waktu perjalanan selama 46 menit.
	\item Durasi waktu tercepat yang dialami penumpang saat melakukan perjalanan adalah hanya 12 menit. Hal ini dapat disebabkan karena jarak tempuh yang dekat, dan bis yang tidak terlalu lama menunggu di stasiun.
	\item Pada kasus terburuk, penumpang dapat menghabiskan waktu hingga sekitar 4 jam di perjalanan. Hal ini dapat disebabkan oleh kombinasi beberapa faktor, diantaranya waktu menunggu bus yang lama, jauhnya jarak tempuh, dan lamanya bus singgah di stasiun pemberhentian.	
\end{enumerate}  

\subsection{Source Code} 
Source code program simulasi ini dapat diakses online di \url{https://github.com/adipurnama141/BusSimulator}

\begin{lstlisting}[language=Python]

import random,sys

def timeFormatter(inputSecond):
	m, s = divmod(inputSecond, 60)
	h, m = divmod(m, 60)
	print "%d:%02d:%02d" % (h, m, s)

def avg(l):
	return reduce(lambda x, y: x + y, l) / len(l)

class Passanger:
	def __init__(self,id,originStation,arrivedTimeAtOriginStation):
		self.id = id
		self.originStation = originStation
		self.targetStation = 0
		self.arrivedTimeAtOriginStation = arrivedTimeAtOriginStation
		self.arrivedTimeAtTargetStation = 0
		self.timeEnteringTheBus = 0
		self.timeLeavingTheBus = 0
		self.delayTimeInQueue = 0
	def enteringTheBus(self, timeEnteringTheBus):
		self.timeEnteringTheBus = timeEnteringTheBus
		self.delayTimeInQueue = self.timeEnteringTheBus - self.arrivedTimeAtOriginStation
	def leavingTheBus(self, timeLeavingTheBus):
		self.timeLeavingTheBus = timeLeavingTheBus

class StationQueue:
	def __init__(self,passangerPerHour,originStation):
		self.futurePassangerQueue = []
		self.currentQueue = []
		self.simulationTime = 0
		self.isFirstStart = True
		self.maxNumberQueue = 0
		self.qtArea_nPeopleOnQueue = 0
		self.initializeAllFutureEvents(passangerPerHour,originStation)
	def initializeAllFutureEvents(self,passangerPerHour,originStation):
		global_time = 0.0
		simulation_end = 81
		simulation_end = simulation_end * 3600
		n_passanger = 0
		stationList = []
		while global_time < simulation_end:
			passanger_arrival = random.expovariate(passangerPerHour) * 3600
			global_time += passanger_arrival
			if (global_time <= simulation_end):
				n_passanger += 1
				m, s = divmod(global_time, 60)
				h, m = divmod(m, 60)
				a = Passanger(n_passanger,originStation,global_time)
				self.futurePassangerQueue.append(a)
	def updateStationQueue(self,time):
		if (self.isFirstStart):
			self.simulationTime = 0
			self.isFirstStart = False
		else:
			while (self.futurePassangerQueue[0].arrivedTimeAtOriginStation <= time):
				passanger = self.futurePassangerQueue.pop(0)
				self.currentQueue.append(passanger)
			self.qtArea_nPeopleOnQueue += len(self.currentQueue) * (time - self.simulationTime)
			self.simulationTime = time
			if (self.maxNumberQueue < len(self.currentQueue)):
				self.maxNumberQueue = len(self.currentQueue)
	def peekFuturePassanger(self):
		return self.futurePassangerQueue[0]


class Bus:
	def __init__(self):
		# Time Related
		self.timeArrivedAtStation = 0
		self.carRentalDepartureTime = 0
		self.simulationTime = 0.0
		# Passangers
		self.passangersInsideBus = []
		self.arrivedPassangers =[]
		self.nOfPassangerFromTerminal = 0
		self.nOfPassangerFromCarRental = 0
		self.nOfPassangerToTerminal1 = 0
		self.nOfPassangerToTerminal2 = 0
		self.nOfPassangerWantToUnloadHere = 0
		# Station Related
		self.StationQueue = dict()
		self.StationQueue[1] = StationQueue(14,1)
		self.StationQueue[2] = StationQueue(10,2)
		self.StationQueue[3] = StationQueue(24,3)
		self.currentPosition = 3
		self.busSpeed = 30 #Miles Per Hour
		#Statistics
		self.busWaitingTimeStatistics = []
		self.loopTimeStatistics = []
		self.qtArea_nPassangerInsideBus = 0
		self.maxNPassangers = 0
		self.isFreshStart = True

	def unloadPassangers(self):
		if (self.nOfPassangerWantToUnloadHere > 0):
			for x in range(self.nOfPassangerWantToUnloadHere):
				passangerWantToUnload = self.passangersInsideBus.pop(0)
				self.unloadPassanger(passangerWantToUnload)

	def loadPassangers(self):
		currentSessionbusWaitingTime = self.getMaxbusWaitingTime()
		currentStationQueue = self.StationQueue[self.currentPosition]
		isWaitingPhase = True
		while (self.simulationTime <= currentSessionbusWaitingTime) and (not self.isFull()) and (isWaitingPhase):

			# Load All Passanger Who Already Waiting in Station Queue as Long Bus is Not Full
			while (len(currentStationQueue.currentQueue) > 0) and (not self.isFull()):
				passanger = currentStationQueue.currentQueue.pop(0)
				self.loadPassanger(passanger)
				# Additional Waiting Time Due People Loading
				if (self.simulationTime > currentSessionbusWaitingTime):
					currentSessionbusWaitingTime = self.simulationTime
			
			# Peek Future Passanger
			# If There is Future Passanger Who Will Come Faster Before WaitTime Ends. Wait...
			# If No. Wait Until WaitTime Ends. Then Go..
			if (currentStationQueue.peekFuturePassanger().arrivedTimeAtOriginStation <= currentSessionbusWaitingTime):
				self.setsimulationTime(currentStationQueue.peekFuturePassanger().arrivedTimeAtOriginStation)
			else:
				isWaitingPhase = False
				# Force Wait Until 5 Minutes (Even Is Predicted There Is No Passanger Will Come Anyway)
				if (self.simulationTime < currentSessionbusWaitingTime):
					self.setsimulationTime(currentSessionbusWaitingTime)
				# Collect The Statistics Data (Waiting Time Duration)	
				busWaitingTime = self.simulationTime - self.timeArrivedAtStation
				self.addbusWaitingTimeStatistics(busWaitingTime)
				# Collect The Statistics Data (Loop Time)
				if ((self.currentPosition == 3) and (self.isFreshStart)):
					self.carRentalDepartureTime = self.simulationTime
				if ((self.currentPosition == 3) and (not self.isFreshStart)):
					loopTime = self.simulationTime - self.carRentalDepartureTime
					self.carRentalDepartureTime = self.simulationTime
					self.addLoopTimeStatistics(loopTime)

	def movePlace(self):
		self.isFreshStart = False	
		# [STATS] Check if Current Number of People in The Bus Breaks Record
		if (self.maxNPassangers < len(self.passangersInsideBus)):
			self.maxNPassangers = len(self.passangersInsideBus)
		if (self.currentPosition == 3):
			self.currentPosition = 1
			self.simulationTime = self.simulationTime + (4.5 / self.busSpeed * 3600)
			self.determinePassangerTargetStation()
			self.nOfPassangerWantToUnloadHere = self.nOfPassangerToTerminal1
			self.nOfPassangerFromTerminal = 0
		elif (self.currentPosition == 1):
			self.currentPosition = 2
			self.simulationTime = self.simulationTime + (1 / self.busSpeed * 3600)
			self.nOfPassangerWantToUnloadHere = self.nOfPassangerToTerminal2
			self.nOfPassangerToTerminal1 = 0
		else:
			self.currentPosition = 3
			self.simulationTime = self.simulationTime + (4.5 / self.busSpeed * 3600)
			self.nOfPassangerWantToUnloadHere = self.nOfPassangerFromTerminal
			self.nOfPassangerToTerminal2 = 0
		self.timeArrivedAtStation = self.simulationTime
		self.timeoutCheck()

	def timeoutCheck(self):
		self.refreshQueue()
		if self.simulationTime >= (80 *3600):
			self.printStatistics()
			sys.exit()

	def refreshQueue(self):
		self.StationQueue[1].updateStationQueue(self.simulationTime)
		self.StationQueue[2].updateStationQueue(self.simulationTime)
		self.StationQueue[3].updateStationQueue(self.simulationTime)

	def printStatistics(self):
		print "Total Passanger Served : " + str(len(self.arrivedPassangers))
		self.printBusWaitingTimeStatistics()
		self.printLoopTimeStatistics()
		self.printNumberPeopleOnBusStatistics()					
		self.printNumberPersonQueueStatistics()
		self.printDelayTimeInQueueStatistics()
		self.printPersonInSystemStatistics()

	def unloadPassanger(self,passanger):
		unloadTime = self.unloadPassangerDuration()
		passanger.timeArrivedAtStation = self.simulationTime
		self.simulationTime += unloadTime
		passanger.timeLeavingTheBus = self.simulationTime
		passanger.targetStation = self.currentPosition
		self.arrivedPassangers.append(passanger)	
		if (passanger.originStation == 3):
			self.nOfPassangerFromCarRental -= 1
		else:
			self.nOfPassangerFromTerminal -= 1
		self.qtArea_nPassangerInsideBus = self.qtArea_nPassangerInsideBus + (self.simulationTime - passanger.timeEnteringTheBus)
		self.timeoutCheck()

	def loadPassanger(self,passanger):
		loadTime = self.loadPassangerDuration()
		passanger.enteringTheBus(self.simulationTime + loadTime)
		self.passangersInsideBus.append(passanger)
		self.simulationTime = passanger.timeEnteringTheBus
		if (passanger.originStation == 3):
			self.nOfPassangerFromCarRental += 1
		else:
			self.nOfPassangerFromTerminal +=1
		self.timeoutCheck()

	def loadPassangerDuration(self):
		rand_var = random.uniform(0,1)
		processed_var = (rand_var * 10) + 15
		return processed_var

	def unloadPassangerDuration(self):
		rand_var = random.uniform(0,1)
		processed_var = (rand_var * 8) + 16
		return processed_var

	def determinePassangerTargetStation(self):
		for x in range(self.nOfPassangerFromCarRental):
			if (random.uniform(0,1) <= 0.583):
				self.nOfPassangerToTerminal1 += 1
			else:
				self.nOfPassangerToTerminal2 += 1

	def printNumberPersonQueueStatistics(self):
		print "Average Number Queue"
		print "	Station1 :",
		print self.StationQueue[1].qtArea_nPeopleOnQueue / self.simulationTime
		print "	Station2 :",
		print self.StationQueue[2].qtArea_nPeopleOnQueue / self.simulationTime
		print "	Station3 :",
		print self.StationQueue[3].qtArea_nPeopleOnQueue / self.simulationTime
		print "Max Number Queue"
		print "	Station1 :",
		print self.StationQueue[1].maxNumberQueue 
		print "	Station2 :",
		print self.StationQueue[2].maxNumberQueue 
		print "	Station3 :",
		print self.StationQueue[3].maxNumberQueue 

	def printBusWaitingTimeStatistics(self):
		l = self.busWaitingTimeStatistics
		avgBusWaitTime = avg(l)
		maxBusWaitTime = max(l)
		minBusWaitTime = min(l)
		print "Bus Waiting Time"
		print "	Max : ",
		timeFormatter(maxBusWaitTime)
		print "	Avg : ",
		timeFormatter(avgBusWaitTime)		
		print "	Min : ",
		timeFormatter(minBusWaitTime)
	
	def printLoopTimeStatistics(self):
		l = self.loopTimeStatistics
		avgLoopTime = avg(l)
		maxLoopTime = max(l)
		minLoopTime = min(l)
		print "Loop Time"
		print "	Max: ",
		timeFormatter(maxLoopTime)
		print "	Avg: ",
		timeFormatter(avgLoopTime)			
		print "	Min: ",
		timeFormatter(minLoopTime)
		
	def printNumberPeopleOnBusStatistics(self):
		#Calculate Area Under Qt Based On Remaining Passanger in The Bus
		for x in range(len(self.passangersInsideBus)):
			self.qtArea_nPassangerInsideBus = self.qtArea_nPassangerInsideBus + (self.simulationTime - self.passangersInsideBus[x].timeEnteringTheBus)
		#Calculate Avg Number of People on The Bus
		avgNumberOfPeopleOnTheBus = self.qtArea_nPassangerInsideBus / self.simulationTime
		print "Number of People On The Bus"
		print "	Max : ",
		print self.maxNPassangers
		print "	Avg : ",
		print avgNumberOfPeopleOnTheBus	

	# Calculate "Time Spent In System" Statistics by Using List of Arrived Passanger
	def printPersonInSystemStatistics(self):
		penumpangInSystemTime = []
		for x in range(len(self.arrivedPassangers)):
			penumpangInSystemTime.append(self.arrivedPassangers[x].timeLeavingTheBus - self.arrivedPassangers[x].arrivedTimeAtOriginStation)
		l = penumpangInSystemTime
		avgInSystemTime = avg(l)
		maxInSystemTime = max(l)
		minInSystemTime = min(l)
		print "Passanger In System Duration"
		print "	Max : ",
		timeFormatter(maxInSystemTime)
		print "	Avg : ",
		timeFormatter(avgInSystemTime)
		print "	Min : ",
		timeFormatter(minInSystemTime)

	# Calculate "Delay Time In Queue" Statistics by Using List of Arrived Passanger
	def printDelayTimeInQueueStatistics(self):
		max_delayTime = [-1,-1,-1]
		sum_delayTime = [0,0,0]
		avg_delayTime = [0,0,0]
		n_passangerInStation  = [0,0,0]
		for x in range(len(self.arrivedPassangers)):
			station_type = self.arrivedPassangers[x].originStation - 1
			n_passangerInStation[station_type] += 1
			sum_delayTime[station_type] += self.arrivedPassangers[x].delayTimeInQueue
			if (self.arrivedPassangers[x].delayTimeInQueue > max_delayTime[station_type]):
				max_delayTime[station_type] = self.arrivedPassangers[x].delayTimeInQueue
		print "Max Delay Time"
		for x in range(0,3):
			avg_delayTime[x] = sum_delayTime[x] / n_passangerInStation[x]		
			print "	Queue " + str(x+1) + " : ",
			timeFormatter(max_delayTime[x])
		print "Avg Delay Time"
		for x in range(0,3):	
			print "	Queue " + str(x+1) + " : ",
			timeFormatter(avg_delayTime[x])

	def addbusWaitingTimeStatistics(self,busWaitingTime):
		self.busWaitingTimeStatistics.append(busWaitingTime)

	def addLoopTimeStatistics(self,loopTime):
		self.loopTimeStatistics.append(loopTime)

	def getMaxbusWaitingTime(self):
		return self.simulationTime + (5 * 60)

	def setsimulationTime(self,newtime):
		self.simulationTime = newtime
		self.timeoutCheck()

	def isFull(self):
		return len(self.passangersInsideBus) >= 20

	def getNPassanger(self):
		return len(self.passangersInsideBus)
	
		
bus = Bus()
while True:
	bus.unloadPassangers()
	bus.loadPassangers()
	bus.movePlace()

 	\end{lstlisting}


\end{document}